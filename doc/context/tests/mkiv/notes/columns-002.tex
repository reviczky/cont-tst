\setupnotes[spacebefore=4*line]
\setupnotes[spaceinbetween=line]

\starttext

    \setupnote [footnote][before=,after=]
    \definenote[toofnote][before=,after=]

    \dorecurse{100}{
        test #1.a\footnote{note #1}
        test #1.b\toofnote{eton #1}
        \par}

    \page

    \setuplanguage[agr][patterns={agr,la}]

    \mainlanguage[agr] % Greek as main language

    \definefallbackfamily [mainface] [serif] [GFS Didot] [preset=range:greek]
    \definefontfamily [mainface] [serif] [TeX Gyre Pagella]

    \setupbodyfont[mainface] % ,7.8pt]

    \setuplayout[header=2cm,footer=2cm]

    \setupnotes[compress=yes]
    \setupnotations[alternative=serried]

    \definenote[aNote]
    \definenote[bNote]
    \definenote[cNote]
    \definenote[dNote]

    \setupnote[aNote][before=,after=]
    \setupnote[bNote][before=,after=]
    \setupnote[cNote][before=,after=]
    \setupnote[dNote][before=,after=]

    \def\ANote#1#2{#1\aNote{#1] #2}}
    \def\BNote#1#2{#1\bNote{#1] #2}}
    \def\CNote#1#2{#1\cNote{#1] #2}}
    \def\DNote#1#2{#1\dNote{#1] #2}}

    \setupalign[hz, hanging]
    \setuptolerance[strict]

\setupnotes[spacebefore=4*line]
\setupnotes[spaceinbetween=]

    % \setuplinenumbering[step=5, location=inright, distance=1ex,align=center, width=0.5em]

    \definemargindata[Stephanus][location=inner, distance=2ex,style=\em]

    % \setupbodyfont[mainface,7.8pt]

    \start\fr % some text in French
    Définir un `apparat critique' et le mettre en page avec un traitement de texte
    courant est un véritable casse-tête. LaTeX et ConTeXt offrent des outils
    d'automatisation encore assez mal connus dans la communauté des éditeurs,
    notamment dans l'édition savante, pour la collation et la comparaison de textes
    médiévaux.
    \stop

    \blank

    \start\en % some text in English
    {\em It is not very easy to define a `criticus apparatus' with some current tools
    (like Microsoft Office Word or LibreOffice). Maybe \ConTeXt offers some ways that
    seem easier, in order to improve clear and precise printing.}
    \stop

    \dorecurse{4}{
        \startmixedcolumns[n=2, balance=yes]
            \Stephanus{1a} Ὁμώνυμα λέγεται ὧν ὄνομα μόνον κοινόν, ὁ δὲ κατὰ τοὔνομα
            λόγος τῆς οὐσίας ἕτερος, οἷον ζῷον ὅ τε ἄνθρωπος καὶ τὸ γεγραμμένον•
            τούτων γὰρ ὄνομα μόνον κοινόν, ὁ δὲ κατὰ τοὔνομα λόγος τῆς οὐσίας ἕτερος•
            ἐὰν γὰρ ἀποδιδῷ τις τί ἐστιν αὐτῶν ἑκατέρῳ τὸ ζῴῳ εἶναι, ἴδιον ἑκατέρου
            λόγον ἀποδώσει. συνώνυμα δὲ λέγεται ὧν τό τε ὄνομα κοινὸν καὶ ὁ κατὰ
            τοὔνομα λόγος τῆς οὐσίας ὁ αὐτός, οἷον ζῷον ὅ τε ἄνθρωπος καὶ ὁ βοῦς•
            τούτων γὰρ ἑκάτερον κοινῷ ὀνόματι προσαγορεύεται ζῷον, καὶ ὁ λόγος δὲ τῆς
            οὐσίας ὁ αὐτός• ἐὰν γὰρ ἀποδιδῷ τις τὸν ἑκατέρου λόγον τί ἐστιν αὐτῶν
            ἑκατέρῳ τὸ ζῴῳ εἶναι, τὸν αὐτὸν λόγον ἀποδώσει.
            \column
            \startlinenumbering[continue]
                Aequivoca dicuntur quorum \CNote{nomen}{first note} solum commune
                est, secundum nomen vero \ANote{substantiae}{second note}
                \ANote{ratio}{second note} diversa, ut animal \DNote{homo}{third
                note} et quod pingitur. Horum enim solum nomen commune est, secundum
                nomen vero substantiae ratio diversa; si enim quis assignet quid est
                utrique eorum quo sint animalia, propriam assignabit utriusque
                rationem. Univoca vero dicuntur quorum et nomen commune est et
                secundum nomen eadem substantiae ratio, ut animal homo atque bos.
            \stoplinenumbering
        \stopmixedcolumns
    }

\stoptext

