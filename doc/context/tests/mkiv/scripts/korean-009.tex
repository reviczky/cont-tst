\starttext

\definefontfeature
  [korean-composed]
  [goodies=hanbatanglvt,
   colorscheme=default,
   decomposehangul=yes,
   mode=node,
   ljmo=yes,
   tjmo=yes,
   vjmo=yes,
   script=hang,
   language=kor]

\definefont[KoreanJMO] [hanbatanglvt*korean-composed]

\startbuffer
한글 자모의 색을 달리하여 출력합니다
한글 자모의 색을 달리하여 출력합니다. \par
완성형을 조합형으로 바꿔주는 과정이 포함되었습니다.\par
한스씨에게 감사드립니다.
\stopbuffer

\definecolor[colorscheme:100:1][r=.75]
\definecolor[colorscheme:100:2][g=.75]
\definecolor[colorscheme:100:3][b=.75]

\definecolor[colorscheme:101:1][g=.75,b=.75]
\definecolor[colorscheme:101:2][r=.75,b=.75]
\definecolor[colorscheme:101:3][r=.75,g=.75]

\startTEXpage[offset=5pt,foregroundstyle=\KoreanJMO]
                            \getbuffer %\par
	\blank
    \setfontcolorscheme[100]\getbuffer
	\blank %\par
    \setfontcolorscheme[101]\getbuffer %\par
\stopTEXpage

\stoptext
