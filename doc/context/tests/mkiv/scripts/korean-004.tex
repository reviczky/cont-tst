% \startluacode
% scripts.installdataset {
%    method  = "hangul",
%    name    = "wider",
% -- parent  = "default",
%    dataset = {
%        inter_char_stretch_factor     = 0.75,
%        inter_char_half_shrink_factor = -0.75,
%     -- inter_char_hangul_penalty     =   50,
%    },
% }
% \stopluacode

\setuplanguage[kr][patterns=us]
\mainlanguage[kr]
\setscript[hangul] % [preset=wider]

\definefontfeature[kr][script=hang,language=kor,mode=node,analyze=yes]
\usetypescriptfile[type-kor-mixed]

\setupbodyfont[unfonts,rm,12pt]

\setupindenting[yes,medium]
\setbreakpoints[compound]

\setuppapersize[B5][B5]
\setuplayout[width=11cm,height=19cm,header=1cm,footer=1cm,marking=on,location=right]

\def\myindex#1{\index{#1}{\bf #1}}

\starttext

\def\showfontparameters
  {\starttabulate[|l|l|]
   \NC slant per point  \NC \the\fontslantperpoint   \font \NC\NR
   \NC interword space  \NC \the\fontinterwordspace  \font \NC\NR
   \NC interword stretch\NC \the\fontinterwordstretch\font \NC\NR
   \NC interword shrink \NC \the\fontinterwordshrink \font \NC\NR
   \NC ex height        \NC \the\fontexheight        \font \NC\NR
   \NC em width         \NC \the\fontemwidth         \font \NC\NR
   \NC extra space      \NC \the\fontextraspace      \font \NC\NR
   \stoptabulate}

\showfontparameters

% fontslantperpoint :0.0pt
% fontinterwordspace :3.912pt
% fontinterwordstretch:1.956pt
% fontinterwordshrink :1.304pt
% fontexheight :5.172pt
% fontemwidth :12.0pt
% fontextraspace :1.304pt

% fontslantperpoint :0.0pt
% fontinterwordspace :2.7pt
% fontinterwordstretch:1.35pt
% fontinterwordshrink :0.9pt
% fontexheight :5.88pt
% fontemwidth :12.0pt
% fontextraspace :0.9pt

\title{소개}

\myindex{조합론}은 \myindex{이산적 구조}를 가진 문제들을 다루는 \myindex{수학}의 한 분야이다. \myindex{연구}의 영역은 어떤 물건들을 미리 정해진 \myindex{조건}에 따라 \myindex{선택하고 배열}하는 일들, 어떤 점들과 그 점들의 \myindex{연결관계}로 이루어진 \myindex{그래프}(\myindex{graph})라고 불리는 어떤 \myindex{구조}에 대한 연구, 그리고 \myindex{특정}한 규칙에 따른 \myindex{실험계획}의 \myindex{디자인} 등을 포함한다. 조합론적 문제들과 그 응용들은 수학의 여러 분야에서만 발견되는 것이 아니라 \myindex{공학}, \myindex{컴퓨터이론}, \myindex{OR}, \myindex{경영과학}, 그리고 \myindex{생명과학}의 분야와 같은 다른 영역에서도 에서도 발견되고 있다.  컴퓨터가 문제들의 이산적 구성을 요구하기 때문에 조합론적 기법이 공학자나 \myindex{응용과학자}들에게, 예를 들면, \myindex{조선사}의 선박 \myindex{건조일정}을 계획하는 일부터 생명과학의 \myindex{인간 유전자} 연구에 이르기까지,  \myindex{필수적}이고 \myindex{강력한 도구}가 되었다.
\blank

\chapter{하나}

특별한 경우에 얼마나 많은 배열이 있는가를 찾는 문제들인, 경우의 수를 세는 문제들은 조합론의 기본적인 문제들 중의 하나이다. \myindex{Counting}은 \myindex{사회과학}에서 \myindex{의사결정} 기관의 (\myindex{주주총회}, \myindex{의회}, 그리고 \myindex{UN 안전보장이사회}와 같은) 참여자들의 \myindex{영향력}을 \myindex{측정}하는 \myindex{샤플리-수빅} (\myindex{Shapley-Shubig}) \myindex{파워 지수}를 계산하는데 이용되기도 하였다. \myindex{화학}에서는 \myindex{케일리}(\myindex{Cayley})가 \myindex{포화탄화수소}의 \myindex{이성질체}의 개수를 세기위해 그래프를 이용하였다; 한편, \myindex{해라리}(\myindex{Harary})와 \myindex{리드}(\myindex{Read})는 \myindex{벤젠 링}으로 부터 만들어지는 어떤 \myindex{유기화합물}을 공유된 변을 따라 연결된 \myindex{육각형}의 \myindex{구조물}로 나타냄으로써  그 개수를 셀 수 있었다. 유전공학에서는 네 개의 \myindex{기본 핵산}들로 구성된 모든 가능한 \myindex{DNA} \myindex{연결구조}를 셈으로써 \myindex{과학자}들은 놀랄만한 큰 수에 \myindex{도달}하게 되었고, 그 결과로 유전자를 구성할 수 있는 엄청나게 많은 \myindex{가능성}을 \myindex{이해}할 수 있게 되었다. Counting은 \myindex{RNA}의 제일, 제이의 구조를 연구하는데 사용되기도 하였다.
\blank
이 책은 \myindex{상급} \myindex{고등학생}들, \myindex{대학} \myindex{초년생}들, 그리고 \myindex{선생님}들에게 기본적인 조합론적 기법을 소개하기위해 쓰여졌다. 또한 이 책이 수학을 즐기는 사람들과 의욕있는 \myindex{퍼즐가}들에게 \myindex{흥미로운 책}이 될 것이라 믿는다.
\blank

\chapter{둘}

이 책의 \myindex{다양한} \myindex{문제}들과 응용들은 counting의 \myindex{능력}을 기르는데 유용한 것만이 아니라 일반적인 \myindex{문제해결} (\myindex{Problem-solving})의 기본 능력과 기법을 \myindex{연마}하는 \myindex{풍부한} \myindex{자료}가 될 것이다. 이 책의 많은 문제들이 \myindex{흔한 경우}들을 \myindex{가급적} 피하고 있어서, 그 결과로 독자들이 그 문제들을 해결하기위해 \myindex{열심히} \myindex{생각}하도록 요구하고 있다. 실제로, \myindex{부지런한} \myindex{독자}는 어떤 \myindex{특정한} 문제를 푸는데 한 가지보다 많은 방법을 발견하곤 하는데, 이것은 \myindex{진정} 문제해결에 있어 \myindex{중요한} \myindex{깨달음}이다. 따라서 이 \myindex{책자}는 \myindex{학생}들이 문제해결의 \myindex{발견적 학습법}과 \myindex{사고 기법} 배우는 이른 \myindex{출발}을 할 수 있도록  돕고있다.
\blank
첫 두 장은 두 가지 \myindex{기본 원리}들, 즉, \myindex{합의 원리}와 \myindex{곱의 원리}를 다루고 있다. 이 두 가지 원리는  Counting에 \myindex{일상적}으로 사용되는데, 자신이 수학을 공부하는 사람이 아니라고 생각하는 사람들에 의해서도 \myindex{사용되고} 있다. \myindex{그렇지만} 이 두 원리가 \myindex{가끔} \myindex{잘못 이해}되고 \myindex{잘못 사용}되기도 한다.

\myindex{언뜻보기}에 \myindex{복잡하게} 보이는 많은 counting 문제들이 단순히 "\myindex{관점의 변화}"를 통해 해결될 수 있다. 5장에서는 이런 \myindex{맥락}에서, 중요한 원리, 즉 \myindex{일대일 대응 원리}를 소개하고 있다; 한편 6장에서는 매우 \myindex{유용한 관점}을 소개하고 있는데 그것은 많은 \myindex{세는 문제}들을 \myindex{상자}에 \myindex{공}을 \myindex{배분}하는 문제들로 바꾸어 생각할 수 있다는 것이다.  그 다음에 나오는 세 장들에서는 많은 응용문제들과 \myindex{변형}들을 통해 \myindex{일대일 대응원리}와  배분 개념에 익숙해지게 하고 있다.
\blank

\chapter{셋}

제 3 장에서 $n\choose r$ 혹은 $C_r^n$로 표현되는 많은 수들을 소개하였다. 마지막 세 장들은 이 수들을 \myindex{이항전개식}과 \myindex{파스칼의 삼각형}을 통해 더 다루고 있다. 많은 \myindex{유용한} \myindex{항등식}들이 \myindex{증명}되었고, 또 이 \myindex{항등식}들이 나타나는 의외의 문제들도 제시되었다.
\blank
\myindex{마지막}으로, 제 \myindex{13 장}에 앞에서 배운 하나 혹은 더 많은 개념들의 응용으로 생각할 수 있는 \myindex{흥미로운} 문제들을 모아 놓음으로써 이 책을 \myindex{마무리}하고 있다. 이 책에 (C)로 표시된 문제들은 \myindex{케임브리지 대학}의 \myindex{Local} \myindex{Examinations} \myindex{Syndicate}의 \myindex{허락}하에, 그리고 (\myindex{AIME})로 표시된 문제들은 \myindex{American} \myindex{Invitational} \myindex{Mathematics} Examination의 허락하에 소개되었다. 이 문제들을 이 책에 소개할 수 있도록 허락해준 두 기관에 감사의 뜻을 표한다.
\blank
이 책은 \myindex{싱가포르} \myindex{수학회}의 \myindex{잡지} 중, \myindex{Matematical Medley}에 처음으로 실린 \myindex{counting}에 대한 \myindex{일련의 글}들 중 \myindex{첫 6개}의 글에 \myindex{기초}를 두고 있다. 원 \myindex{시리즈} \myindex{집필}시 첫 저자를 많이 도와준 \myindex{Tan Ban Pin}에게 감사드린다. 또한 \myindex{초안}을 읽어주고 문제들을 \myindex{확인}해준 우리의 \myindex{동료들}인 \myindex{Dong Fengming}, \myindex{Lee Tuo Yeong}, 그리고 \myindex{Toh Tin Lam}에게도 감사드린다. - 이 책의 어떤 실수도 저자들에게 \myindex{책임}이 있다.
\blank


\myindex{탄광}에서 \myindex{석탄}을 \myindex{캐다가} \myindex{진폐증}에 걸린 사람들이 많이 있는데 그 \myindex{수효}는 정확하게 \myindex{파악되지} 않고 있다. \myindex{쌍수}를 들어 \myindex{환영할} 일이지만 \myindex{깜빡}하다가는 \myindex{큰일}이 날 수도 있다. \myindex{땅}과 \myindex{바다}에 가득하다는 것은 무슨 의미가 있을까?

\chapter{색인}

\placeregister[index]

\stoptext
