\starttext

\setuppapersize[S6][S6]
\setuplayout[width=middle,height=middle,margin=1.5cm,footer=0pt,header=1cm]
\setupcolors[state=start]
\setuptyping[option=color]

\showframe

\title{\type{continue=yes|no}, \type{rule=on|off}}

\subject{Issues: None}

\page

\startbuffer[test]
\setuplinenumbering[continue=no]
\setuplinenote [linenote] [rule=off,frame=on,framecolor=darkred]
\startlinenumbering
\subject{Let's start}
test test test \dorecurse{40}{test }.
\linenote {A simple linenote does not have a number range}
\startlinenote [one] {A linenote environment has a range that covers the
first line of an environment up to the last.}
\dorecurse{40}{test }.
\stoplinenote [one]
\linenote {A simple linenote does not have a number range}
\dorecurse{30}{test }.
\par
\stoplinenumbering
\subject{Continue or not?}
\startlinenumbering
\dorecurse{40}{test }.
\stoplinenumbering
\stopbuffer

{\typebuffer[test] \page \getbuffer[test] \page}

\startbuffer[test]
\setuplinenumbering[continue=yes]
\setuplinenote [linenote] [rule=on,frame=on,framecolor=darkred,n=2]
\startlinenumbering [101]
\subject{Let's start}
test test test \dorecurse{40}{test }.
\linenote {A simple linenote does not have a number range}
\startlinenote [one] {A linenote environment has a range that covers the
first line of an environment up to the last.}
\dorecurse{40}{test }.
\stoplinenote [one]
\linenote {A simple linenote does not have a number range}
\dorecurse{30}{test }.
\par
\stoplinenumbering
\subject{Continue or not?}
\startlinenumbering
\dorecurse{40}{test }.
\stoplinenumbering
\stopbuffer

{\typebuffer[test] \page \getbuffer[test] \page}
\stoptext
