\showframe

\starttext
The general rule (assuming midaligned displayed formulas) should/could be:
\startitemize[n]
\item If the formula is wider than the equation number plus one quad, center the formula and put the tag on the line below.
\item If the centered formula has a distance of at least one quad to the tag, set the formula midaligned.
\item If we are inbetween, set the formula with a distance of one quad to the equation number (this will make it non-centered, but that is life).
\stopitemize

This one goes into the first category:
\placeformula[eq:long]
\startformula
xxxxxxxxxxxx=xxxxxxxxxxxxxxxxxxxxxxxxxxxxxxxxxxxxxxxxxxxxxxxx
\stopformula

This one also goes into the first category:
\placeformula[eq:notsolong]
\startformula
xxxxxxxxxxxx=xxxxxxxxxxxxxxxxxxxxxxxxxxxxxxxxxxxxxxxxxxxxxx
\stopformula

This one goes into the second category:
\placeformula[eq:evenshorter]
\startformula
xxxxxxxxxxxx=xxxxxxxxxxxxxxxxxxxxxxxxxxxxxxxxxxxxx
\stopformula

This one goes into the third category:
\placeformula[eq:evenshorter]
\startformula
xxxxxxxxxxxx=xxxxxxxxxxxxxxxxxxxxxxxxxxxxxxxxxxxxxxxxxx
\stopformula

If the equation numbers are set to the left, it is common practice to put the equation number \emph{above} the equation instead in case it is needed. Except for that, the same rules apply.

\setupformula[location=left]

This one goes into the first category:
\placeformula[eq:long]
\startformula
xxxxxxxxxxxx=xxxxxxxxxxxxxxxxxxxxxxxxxxxxxxxxxxxxxxxxxxxxxxxx
\stopformula

This one also goes into the first category:
\placeformula[eq:notsolong]
\startformula
xxxxxxxxxxxx=xxxxxxxxxxxxxxxxxxxxxxxxxxxxxxxxxxxxxxxxxxxxxx
\stopformula

This one goes into the second category:
\placeformula[eq:evenshorter]
\startformula
xxxxxxxxxxxx=xxxxxxxxxxxxxxxxxxxxxxxxxxxxxxxxxxxxx
\stopformula



This one goes into the third category:
\placeformula[eq:evenshorter]
\startformula
xxxxxxxxxxxx=xxxxxxxxxxxxxxxxxxxxxxxxxxxxxxxxxxxxxxxxxx
\stopformula



\stoptext
