\continuewhenlmtxmode

\starttext

\definemathmatrix [Pmatrix] [matrix:parentheses]
    [align={all:right},simplecommand=Pmatrix]
\definemathmatrix [Tmatrix] [Pmatrix]
    [action=transpose,simplecommand=Tmatrix]
\definemathmatrix [Nmatrix] [Pmatrix]
    [action=negate,simplecommand=Nmatrix]
\definemathmatrix [Xmatrix] [Pmatrix]
    [action={transpose,negate},simplecommand=Xmatrix]
\definemathmatrix [Smatrix] [Pmatrix]
    [action={transpose,{scale,2}},simplecommand=Smatrix]

\startformula
    \Pmatrix{ -1, 2, 3; 4,-5, 6; 7, 8,-9 } \neq
    \Tmatrix{ -1, 2, 3; 4,-5, 6; 7, 8,-9 } \neq
    \Nmatrix{ -1, 2, 3; 4,-5, 6; 7, 8,-9 } \neq
    \Xmatrix{ -1, 2, 3; 4,-5, 6; 7, 8,-9 }
\stopformula

\startformula
    \Pmatrix{ -1, 2, 3; 4,-5, 6; 7, 8,-9 } \neq
    \Smatrix{ -1, 2, 3; 4,-5, 6; 7, 8,-9 }
\stopformula

\startluacode
    mathematics.registersimplematrix("mikael",function(m)
        local s = string.split("mikael")
        for i=1,#m do
            local mi = m[i]
            for j=1,#mi do
                mi[j] = math.random(1,10) * mi[j]
            end
        end
        return m
    end)
\stopluacode

\definemathmatrix [Mmatrix] [Pmatrix]
    [action={mikael},simplecommand=Mmatrix]

\startformula
    \Mmatrix{ -1, 2, 3; 4,-5, 6; 7, 8,-9 } \neq
    \Mmatrix{ -1, 2, 3; 4,-5, 6; 7, 8,-9 }
\stopformula

\stoptext
