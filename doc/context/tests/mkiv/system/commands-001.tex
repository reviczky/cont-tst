% posted by WS on the mailign list ...

\usemodule[setups]

\starttext

When you describe a command, e.g. \type{\startdescription{\cmdbasicsetup[...]} ... \stopdescription}:

\startbuffer [basicsetup]
\cmdbasicsetup [startxtable]
\stopbuffer

\typebuffer [basicsetup]
\getbuffer  [basicsetup]

When you show the syntax of a command without the options:

\startbuffer [shortsetup]
\cmdshortsetup [startxtable]
\stopbuffer

\typebuffer [shortsetup]
\getbuffer  [shortsetup]

When you show the syntax of a command with the options:

\startbuffer [fullsetup]
\cmdfullsetup [startxtable]
\stopbuffer

\typebuffer [fullsetup]
\getbuffer  [fullsetup]

When you want to show the name of a command (similar to \tex{type}):

\startbuffer [internal]
\cmdinternal {startxtable}
\stopbuffer

\typebuffer [internal]
\getbuffer  [internal]

When you want to show the syntax of a command as a float:

\startbuffer [showdefinition]
\showdefinition [startxtable]
\stopbuffer

\typebuffer [showdefinition]
\getbuffer  [showdefinition]

When you want to refer to the definition:

\startbuffer [definition]
\definition [startxtable]
%\definition [startxtable,startembeddedxtable]
\stopbuffer

\typebuffer [definition]
\getbuffer  [definition]

\page

When you have a generated command (e.g. \tex {placefigure}):

\startbuffer [instance]
\cmdbasicsetupinstance {placefloat} {figure}
\cmdshortsetupinstance {placefloat} {figure}
\cmdfullsetupinstance  {placefloat} {figure}
\stopbuffer

\typebuffer [instance]
\getbuffer  [instance]

\stoptext

