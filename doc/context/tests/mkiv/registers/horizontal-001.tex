% this is a variant of an example on the mailing list

\mainlanguage[es]

\setupbodyfont[dejavu]

\defineregister
  [Russian]
  [n=1,
   command=\Words,
   pagenumber=no,
   language=ru,
   textalternative=horizontal,
   distance=0pt]

\setupregister [Russian] [2] [textstyle=bold,left={, }]
\setupregister [Russian] [3] [textstyle=italic,left={, }]

% word category meaning

\setregisterentry [Russian] [entries:1={исчисление},  entries:2={n. neutr.},  entries:3={cálculo}]
\setregisterentry [Russian] [entries:1={исчисление},  entries:2={n. neutr.},  entries:3={cálculo}]
\setregisterentry [Russian] [entries:1={вероятность}, entries:2={n. fem.},    entries:3={probabilidad}]
\setregisterentry [Russian] [entries:1={обозначать},  entries:2={v.},         entries:3={denotar}]
\setregisterentry [Russian] [entries:1={область},     entries:2={n. fem},     entries:3={región, área}]
\setregisterentry [Russian] [entries:1={событие},     entries:2={n. neutr.},  entries:3={evento}]
\setregisterentry [Russian] [entries:1={определение}, entries:2={n. neutr.},  entries:3={definición}]
\setregisterentry [Russian] [entries:1={знание},      entries:2={n. neutr,},  entries:3={conocimiento}]
\setregisterentry [Russian] [entries:1={бесконечно},  entries:2={adv.},       entries:3={infinitamente}]
\setregisterentry [Russian] [entries:1={сборник},     entries:2={n. masc.},   entries:3={colección, compilación, compendio}]
\setregisterentry [Russian] [entries:1={неравенство}, entries:2={n. neutr.},  entries:3={desigualdad}]

\starttext

\starttitle[title=Ruso-español]
    \placeRussian
\stoptitle

\stoptext
