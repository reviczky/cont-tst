% wolfgangs answer to idris question

\defineseparatorset [ParagraphNumber][.,]

\definecounter[ParagraphNumber][numberseparatorset=ParagraphNumber,criterium=all]

\define\ParagraphNumber
  {\incrementcounter[ParagraphNumber]%
   \convertedcounter[ParagraphNumber][numbersegments=1]\quad}

\define\subParagraphNumber
  {\incrementcounter[ParagraphNumber][2]%
   \convertedcounter[ParagraphNumber][numbersegments=1:2]\quad}

\define\subsubParagraphNumber
  {\incrementcounter[ParagraphNumber][3]%
   \convertedcounter[ParagraphNumber][numbersegments=1:3]\quad}

\setupwhitespace[line]

\starttext
    \ParagraphNumber       \input ward  \par
    \subsubParagraphNumber \input klein \par
    \subParagraphNumber    \input ward  \par
    \subsubParagraphNumber \input klein \par
    \subsubParagraphNumber \input ward  \par
    \subParagraphNumber    \input klein \par
    \subsubParagraphNumber \input ward  \par
    \ParagraphNumber       \input klein \par
\stoptext

% % first version, a nice hack, so also shown:
%
% \defineexpandable[1]\PreviousNumber{\number\numexpr#1-1\relax}
%
% \defineconversion[PreviousNumber][\PreviousNumber]
%
% \defineconversionset[ParagraphNumber][n,PreviousNumber,PreviousNumber]
%
% \definecounter[ParagraphNumber][numberconversionset=ParagraphNumber,numberseparatorset=ParagraphNumber]
%
% \define\ParagraphNumber
%   {\incrementcounter[ParagraphNumber]%
%    \incrementcounter[ParagraphNumber][2]%
%    \incrementcounter[ParagraphNumber][3]%
%    \convertedcounter[ParagraphNumber][numbersegments=1]\quad}
%
% \define\subParagraphNumber
%   {\incrementcounter[ParagraphNumber][2]%
%    \incrementcounter[ParagraphNumber][3]%
%    \convertedcounter[ParagraphNumber][numbersegments=1:2]\quad}
%
% \define\subsubParagraphNumber
%   {\incrementcounter[ParagraphNumber][3]%
%    \convertedcounter[ParagraphNumber][numbersegments=1:3]\quad}

