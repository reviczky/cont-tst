\continuewhenlmtxmode

\starttext

\startoverlayMPgraphic{mp:whatever-6}
    fill OverlayBox withcolor "darkgray" ;
    draw matrixbox (1, 1) (2, 1) enlarged OverlayOffset shifted paired(OverlayOffset) withpen pencircle scaled 1pt withcolor "blue" ;
    draw matrixbox (2, 2) (4, 4) enlarged OverlayOffset shifted paired(OverlayOffset) withpen pencircle scaled 1pt withcolor "red" ;
\stopoverlayMPgraphic

\framed
  [synchronize=background,
   align=normal,
   frame=off,
   background=mp:whatever-6,
   backgroundoffset=.5ex,
   foregroundcolor=white]
  {\setmathmatrixanchoring[yes]% left|right|both|yes
   \startmathmatrix
      \NC a_1 \NC a   \NC b   \NC c\NR
      \NC a_2 \NC a   \NC b   \NC c\NR
      \NC a_3 \NC a^2 \NC b   \NC c\NR
      \NC a_4 \NC a   \NC b_2 \NC c\NR
      \NC a_5 \NC a   \NC b   \NC c\NR
      \NC a_6 \NC a   \NC b   \NC c\NR
  \stopmathmatrix}

% \blank

% This can also be done by Alans node module.

\startoverlayMPgraphic{mp:whatever-7}
    fill OverlayBox withcolor "darkgray" ;
    draw lmt_matrix [
        from        = (1, 1),
        to          = (5, 4),
        connect     = { "bottom", "left"  },
        color       = "blue",
        linewidth   = 1pt,
        arrowoffset = 2pt,
        label       = [
            text     = "$\white \scriptscriptstyle x =1$",
            offset   = ExHeight/2,
            anchor   = "lft",
            fraction = .25
        ],
    ] ;
    draw lmt_matrix [
        from       = (4, 1),
        to         = (1, 5),
        connect    = { "bottom", "right" }
        colors     = { "green", "magenta" },
        arrowcolor = "white",
        linewidth  = 1pt,
    ] ;
    draw lmt_matrix [
        cell      = (2, 2),
        shape     = "circle"
        color     = "red",
        linewidth = 1pt,
    ] ;
    draw lmt_matrix [
        cell      = (4, 6),
        shape     = "round",
        radius    = ExHeight/2,
        color     = "red",
        linewidth = 1pt,
    ] ;
    draw lmt_matrix [
        cell      = (2, 6),
        shape     = "path",
        path      = fulldiamond xscaled 3EmWidth yscaled 4ExHeight,
        color     = "red",
        linewidth = 1pt,
    ] ;
    draw lmt_matrix [
        cell      = (2, 6),
        shape     = "scaledpath",
        offset    = .5ExHeight,
        path      = fulldiamond,
        color     = "yellow",
        linewidth = 1pt,
    ] ;
\stopoverlayMPgraphic

\blank

\framed
  [synchronize=background,
   align=normal,
   frame=off,
   offset=overlay,
   background=mp:whatever-7,
   foregroundcolor=white]
  {\setmathmatrixanchoring[both]%
   \startmathmatrix
      \NC a \NC   \NC \quad \NC b \NC   \NR
      \NC   \NC c \NC \quad \NC   \NC   \NR
      \NC   \NC   \NC \quad \NC   \NC   \NR
      \NC   \NC   \NC \quad \NC   \NC d \NR
      \NC e \NC   \NC \quad \NC   \NC   \NR
      \NC   \NC f \NC \quad \NC g \NC   \NR
  \stopmathmatrix}

\blank

\startoverlayMPgraphic{whatever-9}
    fill anchorspan (1, 1) (1, 3) withcolor "blue" ;
    fill anchorspan (2, 1) (3, 3) withcolor "yellow" ;
    fill anchorspan (3, 1) (3, 3) withcolor "green" ;
    path p ; p := anchorcell (2, 2) ;
    fill p withcolor "red" ;
    % now some weird stuff:
    picture q ; q := externalfigure "t:/sources/cow.pdf" ;
    q := q xysized (bbwidth(p), bbheight(p)) ;
    draw q shifted - center q shifted center p ;
    % and some even weirder stuff:
    \includeMPoverlaydata
\stopoverlayMPgraphic

\enabletrackers[localanchor]

\bTABLE[frame=off,synchronize=background,background=whatever-9]
    \bTR
        \bTD[foregroundcolor=white]
            \startMPoverlaydata
                fill anchorcell (\xanchor,\yanchor) enlarged -.25ExHeight withcolor "darkgray" ;
            \stopMPoverlaydata
            test
        \eTD
        \bTD
            test test
        \eTD
        \bTD
            test test test
        \eTD
    \eTD
    \bTR
        \bTD[foregroundcolor=white]
            test test
        \eTD
        \bTD[nx=2,align=middle,foregroundstyle=bold,foregroundcolor=white]
            test
        \eTD
    \eTD
    \bTR
        \bTD[foregroundcolor=white]
            test test test
        \eTD
        \bTD
            test test
        \eTD
        \bTD
            test
        \eTD
    \eTD
\eTABLE

\stoptext

